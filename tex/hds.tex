\documentclass[11pt,a4paper]{article}

% =======================
% Packages
% =======================
\usepackage[utf8]{inputenc}
\usepackage[T1]{fontenc}
\usepackage[french]{babel}
\usepackage{amsmath, amssymb}
\usepackage{algorithm2e}
\usepackage{geometry}
\usepackage{hyperref}

\geometry{margin=2.5cm}

% =======================
% Document
% =======================
\begin{document}

\title{Algorithme HDS pour le Problème du Voyageur de Commerce}
\author{Mohamed Elakef Zenagui}
\date{}
\maketitle

\section{Introduction}

Le problème du voyageur de commerce (\textit{Travelling Salesman Problem}, TSP) consiste à déterminer une tournée hamiltonienne
de coût minimal passant exactement une fois par chaque sommet d’un graphe complet valué.
Ce problème est NP-difficile.

Dans ce travail, nous présentons un algorithme exact basé sur le principe du \textit{Branch and Bound},
utilisant une borne inférieure obtenue par l’heuristique de la demi-somme, noté \textbf{HDS}.

\section{Structures de données}

CONSTANTE \textbf{n} = 10\\
\newline
TYPE \textbf{Arrete} = enregistrement\\
\ \ $sommet$ : entier \\
\ \ $suiv$ : $\uparrow$ Arrete \\
FIN\\
\newline
TYPE \textbf{GrapheM} = enregistrement\\
\ \ n : entier\\
\ \ M : Tableau[1..n][1..n] de \textbf{réel}\\
FIN\\
\newline
TYPE \textbf{NoeudTas} = enregistrement\\
\ \ borne : réel\\
\ \ cout : réel\\
\ \ cycle : $\uparrow$ Arrete\\
\ \ suiv : $\uparrow$ NoeudTas\\
FIN\\

TYPE \textbf{Tas} = $\uparrow \textbf{NoeudTas}$\\
\newline

\section{Heuristique de la demi-somme}

L’heuristique de la demi-somme fournit une borne inférieure du coût restant à parcourir.
Pour chaque sommet on considère la demi-somme des deux plus petites arêtes incidentes admissibles.

\begin{algorithm}[H]
\caption{Heuristique de la demi-somme}
\KwIn{GrapheM $G$, $\uparrow Arrete$ $c$}
\KwOut{Entier borne}
\textbf{Var} k, i : Entier\\
\ \ \ \ \ \ distances : Tableau[1..n] de réels\\
\ \ \ \ \ \ p, q, r : $\uparrow Arrete$\\  
$borne \leftarrow 0$ \\
$k \leftarrow |c|$ \\

\For{$i \leftarrow 1$ \KwTo $G.n$}{
    \If{$k = 1\ \textbf{or}\ \textbf{not}\ i \in sommets(c)$}{
        $distances \leftarrow tri(G.M[i])$ \\
        $borne \leftarrow borne + distances[2] + distances[3]$ \\
    }
}

$p \leftarrow c$ \\
$q \leftarrow p \uparrow suiv$ \\
\While{$q \neq \text{NIL}$}{
    \If{p = c \textbf{ou}\ $q \uparrow suiv = \text{NIL}$}{
    		$borne \leftarrow borne + G.M[p \uparrow sommet, q \uparrow sommet]$ \\
    		\If{$q \uparrow suiv = \text{NIL}$}{
    			$distances \leftarrow tri(G.M[q \uparrow sommet])$ \\
    		}
    		\Else{
    			$distances \leftarrow tri(G.M[p \uparrow sommet])$ \\
    		}	
    		\If{$distances[2] = G.M[p \uparrow sommet, q \uparrow sommet]$}{
    			$borne \leftarrow borne + distances[3]$
    		}
    		\Else{
    			$borne \leftarrow borne + distances[2]$
    		}
    }
    \Else{
		$r \leftarrow q \uparrow suiv$ \\
		$borne \leftarrow borne + G.M[p \uparrow sommet, q \uparrow sommet] + G.M[q \uparrow sommet, r \uparrow sommet]$ \\ 
    }
    $p \leftarrow q$ \\
    $q \leftarrow q \uparrow suiv$ \\
}

\Return $\frac{borne}{2}$
\end{algorithm}

\section{Algorithme HDS}

L’algorithme HDS explore l’espace des solutions partielles en privilégiant les nœuds possédant la plus petite borne inférieure.

\begin{algorithm}[H]
\caption{Algorithme HDS}
\KwIn{GrapheM $G$}
\KwOut{$\uparrow Arrete$ $best\_solution$, Réel $best\_cost$}
\textbf{Var} u, v, n : Entier\\
\ \ \ \ \ \ borne, h : Réel\\
\ \ \ \ \ \ T : Tas\\
\ \ \ \ \ \ c, $new\_c$ : $\uparrow Arrete$\\

$n \leftarrow |M|$ \\
$best\_cost \leftarrow +\infty$ \\
$best\_solution \leftarrow \text{NIL}$ \\
$T \leftarrow \emptyset$ \\
$c \leftarrow \emptyset$ \\
ajouter(c, 1)\\
$cost \leftarrow 0$ \\

$borne \leftarrow$ Heuristique\_Demi\_Somme$(M, c)$ \\
\textbf{Entasser}(T, $\langle borne, cost, c \rangle$)\\

\While{$T \neq \text{NIL}$}{
    \textbf{Détasser}(T, $\langle borne, cost, c \rangle$)\\
    \If{$borne < best\_cost$}{
    
    \If{$|c| = n$}{
        $total\_cost \leftarrow cost + M[\text{dernier}(c)][1]$ \\
        \If{$total\_cost < best\_cost$}{
            $best\_cost \leftarrow total\_cost$ \\
            $best\_solution \leftarrow c$
        }
    }
    \Else{
    $u \leftarrow$ dernier(c) \\
    \For{$v \leftarrow 1$ \KwTo $n$}{
        \If{$\textbf{not}\ v \in sommets(c)$}{
            $new\_cost \leftarrow cost + M[u][v]$ \\
            $new\_c \leftarrow c$ \\
            ajouter($new\_c$, v) \\
            $h \leftarrow$ Heuristique\_Demi\_Somme$(M,\ new\_c)$ \\
            \If{$h < best\_cost$}{
                \textbf{Entasser}(T,\  $\langle h, new\_cost,\ new\_c \rangle$)
            }
        }
    }
    }
    }
}

\Return $(best\_solution, best\_cost)$
\end{algorithm}

\section{Conclusion}

L’algorithme HDS est un algorithme exact pour la résolution du TSP.
L’utilisation de l’heuristique de la demi-somme permet d’obtenir une borne inférieure admissible,
réduisant considérablement l’espace de recherche par élagage.

\end{document}