\documentclass[11pt,a4paper]{article}

% =======================
% Packages
% =======================
\usepackage[utf8]{inputenc}
\usepackage[T1]{fontenc}
\usepackage[french]{babel}
\usepackage{geometry}
\usepackage{listings}
\usepackage{xcolor}
\usepackage{hyperref}

\geometry{margin=2cm}

% =======================
% Paramètres pour le code
% =======================
\lstset{
    language=Python,
    basicstyle=\ttfamily\small,
    keywordstyle=\color{blue}\bfseries,
    commentstyle=\color{green!50!black},
    stringstyle=\color{red},
    showstringspaces=false,
    numbers=left,
    numberstyle=\tiny,
    frame=single,
    breaklines=true
}

% =======================
% Titre
% =======================
\title{Notice d'utilisation de la classe PVC\_points}
\author{ZENAGUI MOHAMEDELAKEF}
\date{\today}

\begin{document}

\maketitle

\section{Introduction}
La classe \texttt{PVC\_points} permet de résoudre le \textbf{Problème du Voyageur de Commerce (PVC)} sur un ensemble de points 2D.  
Elle fournit plusieurs algorithmes pour générer un cycle optimal ou approché :

\begin{itemize}
    \item \textbf{PPP} : Point le plus proche.
    \item \textbf{OptPPP} : PPP optimisé avec l'algorithme 2-opt.
    \item \textbf{OptPrim} : Arbre couvrant minimal + parcours DFS pour générer un cycle.
    \item \textbf{HDS} : Heuristique de la demi-somme (branch-and-bound).
\end{itemize}

\section{Installation}
Pour utiliser la classe, installer les packages nécessaires :
\begin{lstlisting}
pip install numpy matplotlib
\end{lstlisting}

\section{Création d'un objet PVC\_points}
\begin{lstlisting}
from src.PVC_points import PVC_points

pvc = PVC_points()
\end{lstlisting}

\section{Chargement ou génération des points}

\subsection{Depuis une liste de points}
\begin{lstlisting}
points = [(0,0), (1,2), (3,1), (4,5)]
pvc.charger_de_liste(points)
\end{lstlisting}

\subsection{Depuis un fichier texte}
Le fichier doit contenir un point par ligne au format \texttt{(x, y)} :
\begin{lstlisting}
(0,0)
(1,2)
(3,1)
(4,5)
\end{lstlisting}
\begin{lstlisting}
pvc.charger_de_fichier("data/points.txt")
\end{lstlisting}

\subsection{Générer des points aléatoires en mémoire}
\begin{lstlisting}
points_alea = PVC_points.generer_points(10)  # 10 points aleatoires
pvc.charger_de_liste(points_alea)
\end{lstlisting}

\subsection{Générer un fichier de points aléatoires}
\begin{lstlisting}
PVC_points.generer_fichier(10, nom="points_alea")
# cree ./data/points_alea.txt
\end{lstlisting}

\section{Algorithmes disponibles}
Les méthodes utilisent un \textbf{point de départ aléatoire} défini lors du chargement des points.

\subsection{PPP - Point le plus proche}
\begin{lstlisting}
cycle_ppp = pvc.PPP()
print("Cycle PPP :", cycle_ppp)
\end{lstlisting}

\subsection{OptPPP - PPP optimisé 2-opt}
\begin{lstlisting}
cycle_opt = pvc.OptPPP()
print("Cycle OptPPP :", cycle_opt)
\end{lstlisting}

\subsection{OptPrim - Arbre couvrant minimal + DFS}
\begin{lstlisting}
cycle_prim = pvc.OptPrim()
print("Cycle OptPrim :", cycle_prim)
\end{lstlisting}

\subsection{HDS - Heuristique demi-somme}
\begin{lstlisting}
cycle_hds = pvc.HDS()
print("Cycle HDS :", cycle_hds)
\end{lstlisting}

\section{Longueur d'un cycle}
\begin{lstlisting}
longueur = pvc.longueur(cycle_ppp)
print("Longueur du cycle PPP :", longueur)
\end{lstlisting}

\section{Tracer un cycle}
\begin{lstlisting}
PVC_points.plot_cycle(cycle_ppp, titre="Cycle PPP")
PVC_points.plot_cycle(cycle_opt, titre="Cycle OptPPP")
PVC_points.plot_cycle(cycle_hds, titre="Cycle HDS")
\end{lstlisting}

\section{Exemple complet}
\begin{lstlisting}
from src.PVC_points import PVC_points

points = PVC_points.generer_points(15)
pvc = PVC_points()
pvc.charger_de_liste(points)

cycle_ppp = pvc.PPP()
cycle_opt = pvc.OptPPP()
cycle_hds = pvc.HDS()

print("PPP :", cycle_ppp)
print("OptPPP :", cycle_opt)
print("HDS :", cycle_hds)

print("Longueur PPP :", pvc.longueur(cycle_ppp))
print("Longueur OptPPP :", pvc.longueur(cycle_opt))
print("Longueur HDS :", pvc.longueur(cycle_hds))

PVC_points.plot_cycle(cycle_ppp, "Cycle PPP")
PVC_points.plot_cycle(cycle_opt, "Cycle OptPPP")
PVC_points.plot_cycle(cycle_hds, "Cycle HDS")
\end{lstlisting}

\section{Conseils d'utilisation}
\begin{itemize}
    \item Le nombre minimal de points pour PPP, OptPPP et HDS est \textbf{3}.
    \item L'algorithme HDS peut être très lent pour \textbf{plus de 12 points}.
    \item La variable \texttt{seed\_} garantit un point de départ fixe pour PPP et OptPPP si les points ne sont pas rechargés.
    \item Les cycles sont automatiquement fermés lors du tracé pour visualiser la tournée complète.
\end{itemize}

\end{document}
